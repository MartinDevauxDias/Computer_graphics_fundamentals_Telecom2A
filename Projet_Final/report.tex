\documentclass[11pt,a4paper]{article}

% --- Packages ---
\usepackage[utf8]{inputenc}
\usepackage[T1]{fontenc}
\usepackage[english]{babel}
\usepackage{amsmath, amsfonts, amssymb}
\usepackage{graphicx}
\usepackage{hyperref}
\usepackage{listings}
\usepackage{xcolor}
\usepackage{geometry}
\usepackage{cite}
\usepackage{booktabs}

\geometry{margin=2.5cm}

% --- Code Styling ---
\definecolor{codegreen}{rgb}{0,0.6,0}
\definecolor{codegray}{rgb}{0.5,0.5,0.5}
\definecolor{codepurple}{rgb}{0.58,0,0.82}
\definecolor{backcolour}{rgb}{0.95,0.95,0.92}

\lstset{
    backgroundcolor=\color{backcolour},
    commentstyle=\color{codegreen},
    keywordstyle=\color{magenta},
    numberstyle=\tiny\color{codegray},
    stringstyle=\color{codepurple},
    basicstyle=\ttfamily\footnotesize,
    breakatwhitespace=false,
    breaklines=true,
    captionpos=b,
    keepspaces=true,
    numbers=left,
    showspaces=false,
    showstringspaces=false,
    showtabs=false,
    tabsize=2,
    frame=single
}

% --- Title Information ---
\title{Final Project \\ \large IGR Project Report}
\author{Martin --- Telecom Paris}
\date{\today}

\begin{document}

\maketitle

\begin{abstract}
Implementation of a real-time rendering engine with rigid-body dynamics and a Monte Carlo path tracer.
\end{abstract}

\tableofcontents
\newpage

\section{Introduction}
The objective of this project was to develop a renderer capable of different geometry, rigid body dynamics while maintaining a capacity for raytracing the different scenes. This engine utilizes a path-tracing approach to resolve global illumination, including soft shadows, reflections, and refractions.

\section{Project Evolution and Implementation Timeline}
The development of this project was separated into different steps, focusing on a different aspect of the rendering.

\subsection{Phase 1: Geometry and Subdivision}
Early implementation focused on loading simple meshes and performing subdivision to increase geometric detail. I implemented the Loop subdivision scheme, which defines new vertex positions based on a weighted average of adjacent vertices.

\begin{equation}
v_{new} = (1 - n\beta) v_{old} + \beta \sum_{i=1}^{n} v_i
\end{equation}

Here, $v_{old}$ is the original vertex position, $n$ represents the valence (number of connected edges) of the vertex, $v_i$ are the positions of the neighboring vertices, and $\beta$ is a weighting factor derived from $n$.

\subsection{Phase 2: Physics Foundations}
Initial physics experiments utilized a basic vertex-based solver. The original model simulated collisions by checking if any vertex of a cube intersected the ground plane ($y=0$). 
\begin{itemize}
    \item \textbf{Early Model}: Simple force restitution without friction. It resulted in "bouncy" but unrealistic behavior where objects would rotate incorrectly or pass through each other.
    \item \textbf{Collision Challenges}: Implementing object-to-object collisions introduced significant jitter and "energy gain," where objects would shake until they exploded out of the scene.
    \item \textbf{Current Solution}: To solve this, I moved to the impulse-based solver described in Section 2. I also implemented \texttt{\#pragma omp parallel for} to parallelize the collision detection phase. This allowed me to run the simulation at very high frequencies (small $dt$) with multiple sub-steps per frame, ensuring stability even during complex interactions.
\end{itemize}

\subsection{Phase 3: From Raytracing to Path Tracing}
The rendering pipeline underwent a major refactor after the first prototype.
\begin{itemize}
    \item \textbf{Prototype}: A simple recursive raytracer that handled a single bounce (Only the reflection of a material could be changed).
    \item \textbf{Architectural Refactor}: The code was restructured into a modular hierarchy:
    \begin{itemize}
        \item \texttt{Renderer}: Handles the OpenGL context and Compute Shader dispatching.
        \item \texttt{Scene}: Manages the collection of objects and light sources.
        \item \texttt{RigidSolver}: Encapsulates the physics logic, independent of the rendering.
        \item \texttt{Mesh/Object}: Handles geometry and material properties.
    \end{itemize}
\end{itemize}

\begin{itemize}
    \item \textbf{Monte Carlo Transition}: The final phase involved implementing the full Monte Carlo Path Tracer. This added support for diffuse inter-reflections, soft shadows, and the complex glare effects discussed in Section 5.
\end{itemize}

\begin{figure}[h]
    \centering
    \includegraphics[width=0.8\textwidth]{screenshots/Main Scene.png}
    \caption{The engine's main interface showing various materials (Gold, Glass, Matte) reacting to a central light source.}
\end{figure}

\section{Physics Engine Architecture}
The physics module is based on a rigid-body solver using a discrete-time integration scheme.

\subsection{Numerical Integration}
I employ a semi-implicit Euler integration for both linear and angular components. The state of each object is defined by its position $\mathbf{x}$, velocity $\mathbf{v}$, orientation quaternion $\mathbf{q}$, and angular velocity $\omega$.

\begin{equation}
\mathbf{v}_{t+dt} = \mathbf{v}_t + \frac{\mathbf{F}_{net}}{m} dt, \quad \mathbf{x}_{t+dt} = \mathbf{x}_t + \mathbf{v}_{t+dt} dt
\end{equation}

In these equations, $\mathbf{F}_{net}$ is the sum of all forces acting on the body, $m$ is the mass, and $dt$ is the discrete time step. For rotations, the orientation is updated using the angular velocity:
\begin{equation}
\mathbf{q}_{t+dt} = \text{normalize}\left( \mathbf{q}_t + \frac{1}{2} (\omega \otimes \mathbf{q}_t) dt \right)
\end{equation}

where $\otimes$ denotes quaternion multiplication. The integration step is parallelized using OpenMP to handle large object counts efficiently:

\subsection{Temporal Stability Mechanisms}
Achieving stable simulation with semi-implicit Euler integration required several complementary stabilization techniques, each targeting a distinct failure mode.

\subsubsection*{Velocity Damping}
A small multiplicative damping factor is applied to both linear and angular velocities at each step:
\begin{equation}
\mathbf{v}_{t+dt} \leftarrow 0.999 \cdot \mathbf{v}_{t+dt}, \quad \boldsymbol{\omega}_{t+dt} \leftarrow 0.999 \cdot \boldsymbol{\omega}_{t+dt}
\end{equation}
This dissipates the artificial energy that discrete integration tends to accumulate over time. Without it, objects in sustained contact slowly drift and eventually explode out of the scene due to numerical energy gain.

\subsubsection*{Baumgarte Stabilization}
Penetration errors are corrected by adding a velocity bias term to each contact constraint. Rather than correcting position directly (which can cause instability), the Baumgarte scheme feeds the positional error back as a target velocity:
\begin{equation}
b = \frac{\beta}{\Delta t} \max(0,\ d - d_{slop})
\end{equation}
where $\beta = 0.10$ is the correction factor, $d$ is the measured penetration depth, and $d_{slop} = 0.01\,\text{m}$ is a tolerance threshold. The slop prevents micro-collisions from generating corrective impulses, which would otherwise cause resting objects to jitter continuously on flat surfaces.

\subsubsection*{Quaternion Renormalization}
Repeated quaternion multiplication accumulates floating-point drift, causing the orientation $\mathbf{q}$ to deviate from unit length. This manifests as a gradual distortion of the inertia tensor in world space. After each angular integration step, I explicitly renormalize:
\begin{equation}
\mathbf{q}_{t+dt} = \frac{\mathbf{q}_{t+dt}}{\|\mathbf{q}_{t+dt}\|}
\end{equation}
The world-space inverse inertia tensor $\mathbf{I}_w^{-1} = R\, \mathbf{I}_b^{-1} R^T$ is then recomputed from the corrected rotation matrix $R$, keeping it physically consistent at every frame.


\subsection{Collision Detection and Response}
The solver implements an impulse-based model for collision response. When a collision is detected at a contact point with normal $\mathbf{n}$, the required impulse $J$ to satisfy the law of restitution $e$ is calculated as:

\begin{equation}
J = \frac{-(1 + e) \mathbf{v}_{rel} \cdot \mathbf{n}}{\mathbf{n} \cdot \mathbf{n} (\frac{1}{m_A} + \frac{1}{m_B}) + (\mathbf{I}_A^{-1}(\mathbf{r}_A \times \mathbf{n}) \times \mathbf{r}_A + \mathbf{I}_B^{-1}(\mathbf{r}_B \times \mathbf{n}) \times \mathbf{r}_B) \cdot \mathbf{n}}
\end{equation}  

Here, $m_A$ and $m_B$ are the masses of the colliding bodies, $\mathbf{I}_A^{-1}$ and $\mathbf{I}_B^{-1}$ are their respective inverse inertia tensors in world space, $\mathbf{r}_A$ and $\mathbf{r}_B$ are the vectors from the center of mass to the contact point, and $\mathbf{v}_{rel}$ is the relative velocity at the point of impact.

\section{Vertex Animation and GPU Synchronization}
One other feature of the engine is the support for dynamic, CPU-animated meshes within a ray-tracing context. In Scene 5 (Sea Scene), water waves are created CPU side and synchronize the vertex data with the GPU every frame.

\subsection{Wave Synthesis}
I used a sum of Gerstner-like sinusoids to perturb the grid height. To give the water a "rippled" or "peaky" appearance, I utilize an absolute sine formulation:
\begin{lstlisting}[language=C++]
float h = 0.0f;
h += 0.4f * (1.0f - std::abs(std::sin(0.15f * x + 1.2f * t)));
h += 0.25f * (1.0f - std::abs(std::sin(0.12f * z + 0.8f * t + 1.0f)));
h += 0.15f * (1.0f - std::abs(std::sin(0.08f * (x + z) + 1.5f * t)));
\end{lstlisting}

\subsection{Data Synchronization (SSBOs)}
Because the Compute Shader needs current triangle positions for intersection testing,  the Shader Storage Buffer Objects (SSBO) must be updated. This is handled by mapping the vertex data and re-calculating normals on the CPU before a \texttt{glBufferSubData} call.

\section{GPU Path Tracing Pipeline}
Rendering is offloaded to the GPU via OpenGL 4.5 Compute Shaders. All scene geometry, including mesh triangles and object properties, is uploaded to SSBOs.

\subsection{Monte Carlo Integration}
The renderer solves the rendering equation by sampling paths through the scene. For each pixel, multiple rays are jittered for anti-aliasing. The governing equation for the outgoing radiance $L_o$ is:

\begin{equation}
L_o(\mathbf{p}, \omega_o) = \int_{\Omega} L_i(\mathbf{p}, \omega_i) f_r(\mathbf{p}, \omega_i, \omega_o) \cos \theta_i d\omega_i
\end{equation}

where $\mathbf{p}$ is the hit point, $\omega_o$ the outgoing direction, $L_i$ the incoming radiance from direction $\omega_i$, $f_r$ the Bidirectional Reflectance Distribution Function (BRDF), and $\Omega$ the hemisphere of possible incoming light directions. To handle the hemisphere sampling required for diffuse reflections, I implemented a cosine-weighted distribution.

\begin{lstlisting}[language=GLSL]
// Cosine-weighted hemisphere sampling
vec3 randomInHemisphere(vec3 normal) {
    float u1 = random();
    float u2 = random();
    float r = sqrt(u1);
    float theta = 2.0 * 3.14159 * u2;
    vec3 localDir = vec3(r * cos(theta), r * sin(theta), sqrt(max(0.0, 1.0 - u1)));
    
    vec3 up = abs(normal.z) < 0.999 ? vec3(0,0,1) : vec3(1,0,0);
    vec3 tangent = normalize(cross(up, normal));
    vec3 bitangent = cross(normal, tangent);
    return normalize(tangent * localDir.x + bitangent * localDir.y + normal * localDir.z);
}
\end{lstlisting}

\subsection{Ray-Triangle Intersection}
I then utilized the Möller-Trumbore algorithm for its efficiency and low memory footprint, as it avoids storing plane equations for every triangle.

\begin{lstlisting}[language=GLSL]
bool intersectTriangle(Ray ray, vec3 v0, vec3 v1, vec3 v2, out float t) {
    vec3 edge1 = v1 - v0, edge2 = v2 - v0;
    vec3 h = cross(ray.direction, edge2);
    float a = dot(edge1, h);
    if (a > -0.00001 && a < 0.00001) return false;
    float f = 1.0 / a;
    vec3 s = ray.origin - v0;
    float u = f * dot(s, h);
    if (u < 0.0 || u > 1.0) return false;
    vec3 q = cross(s, edge1);
    float v = f * dot(ray.direction, q);
    if (v < 0.0 || u + v > 1.0) return false;
    t = f * dot(edge2, q);
    return t > 0.00001;
}
\end{lstlisting}

\subsection{Temporal Accumulation}
To reduce the variance (noise) without skyrocketing the per-frame sample count, I implemented a temporal accumulation buffer. If the scene is static, each new frame is averaged with previous ones, effectively increasing the samples per pixel over time.

\begin{lstlisting}[language=GLSL]
vec3 finalColor = currentFrameColor;
if (frameCounter > 1) {
    vec3 prevColor = imageLoad(accumulationBuffer, texelCoord).rgb;
    // Iterative average
    finalColor = mix(prevColor, currentFrameColor, 1.0 / float(frameCounter));
}
imageStore(accumulationBuffer, texelCoord, vec4(finalColor, 1.0));
\end{lstlisting}

\subsection{Fresnel and Dielectric Materials}
To simulate water and glass, I used the Schlick approximation for the Fresnel coefficient $R(\theta)$:
\begin{equation}
R(\theta) = R_0 + (1 - R_0)(1 - \cos \theta)^5, \quad R_0 = \left( \frac{n_1 - n_2}{n_1 + n_2} \right)^2
\end{equation}

where $\theta$ is the angle of incidence, and $n_1, n_2$ are the refractive indices of the two media. Additionally, I implement Beer-Lambert absorption for rays traveling inside transparent volumes:
\begin{equation}
I(\text{dist}) = I_0 \cdot e^{-\sigma \cdot \text{dist}}
\end{equation}

where $I_0$ is the initial light intensity, $\sigma$ is the absorption coefficient (related to the material color), and \textit{dist} is the distance traveled within the medium.

\section{Optical Effects and Challenges}

\subsection{Volumetric Glare Approximation}
A difficulty I still haven't resolved was simulating atmospheric scattering (glare) without the heavy cost of volumetric path marching. I implemented a power-law approximation based on the angular distance between the ray direction $\mathbf{d}$ and the vector to the light $\mathbf{l}$. 

This effect simulates the "glow" seen around strong light sources in a humid or dusty environment. By separating the core intensity from the atmospheric halo, the Ray-tracer can create a spherical illumination volume even with purely angular math.

\begin{lstlisting}[language=GLSL]
// Simplified Glare Loop logic
for (int i = 0; i < objectCount; i++) {
    if (objects[i].emissive.w > 0.0) {
        float dotL = dot(ray.direction, dirToLight);
        if (dotL > 0.0) {
            float intensity = pow(dotL, 4096.0) * 2.0; // Sharp Light Core
            float halo = pow(dotL, 128.0) * 0.07;      // Diffuse Atmospheric Halo
            sampleColor += throughput * lightColor * (intensity + halo);
        }
    }
}
\end{lstlisting}

\begin{figure}[h]
    \centering
    \includegraphics[width=0.7\textwidth]{screenshots/Glare.png}
    \caption{Testing atmospheric scattering using angular distance. The glare intensity follows a power-law distribution to simulate light bleeding.}
\end{figure}

\subsection{Implementation Hardships}
\textbf{Physics Fine-tuning:} Stability was the greatest challenge. Small errors in the inertia tensor or the collision epsilon caused objects to "jitter" or explode. Bypassing the solver for animated meshes (like the sea) was necessary to prevent unwanted collision resolution between fixed planes and moving vertices.

\textbf{Noise and Convergence:} As a Monte Carlo-based system, the engine suffers from high-frequency noise. While I implemented accumulation, it requires low movement to converge. Reducing the triangle count for the sea grid (grid resolution of 5x5) was an essential optimization to maintain frame rates above 30 FPS.

\section{Scene Gallery and Visual Analysis}
To evaluate the performance and correctness of the engine, I designed five distinct scenes, each targeting a specific subset of the system's capabilities.

\subsection{Scene 1: Rigid Body Stress Test}
This scene is dedicated to validating the physics solver. It consists of a $4\times4\times4$ "cube of cubes" (64 individual rigid bodies) stacked in a grid.
\begin{itemize}
    \item \textbf{Objective}: Test stable stacking and multi-object collision resolution.
    \item \textbf{Interaction}: The user can "shoot" high-velocity spheres into the structure to observe the impulse propagation and subsequent collapse.
    \item \textbf{Key Result}: Demonstrates the stability of the impulse-based solver and the efficiency of the OpenMP-parallelized collision loops.
\end{itemize}

\subsection{Scene 2: Material Properties and Monte Carlo Noise}
A showcase of the PBR material system, featuring various spheres (Gold, Silver, Glass, and Matte) set before a large mirror plane and illuminated by an artificial sun.
\begin{itemize}
    \item \textbf{Objective}: Visualize the difference between specular and rough surface transport.
    \item \textbf{Visual Analysis}: This scene highlights the "noise" or "grain" inherent in Monte Carlo integration, particularly on rough materials where the probability density function (PDF) must sample a wider hemisphere.
\end{itemize}

\begin{figure}[h]
    \centering
    \includegraphics[width=0.45\textwidth]{screenshots/Roughness Noise.png}
    \hfill
    \includegraphics[width=0.45\textwidth]{screenshots/Roughness.png}
    \caption{Left: Monte Carlo noise visible during early convergence on rough surfaces. Right: Converged materials after temporal accumulation.}
\end{figure}

\subsection{Scene 3: Recursive Mirror Room}
A single mesh object placed inside a "box" where all walls are perfect mirrors. 
\begin{itemize}
    \item \textbf{Objective}: Study the effect of recursion depth (ray bounces) on global illumination.
    \item \textbf{Comparison}: By toggling between 2 bounces and 12 bounces, one can see the "hall of mirrors" effect emerge as light paths are allowed to travel further through the scene before termination.
\end{itemize}

\begin{figure}[h]
    \centering
    \includegraphics[width=0.45\textwidth]{screenshots/5 bounces.png}
    \hfill
    \includegraphics[width=0.45\textwidth]{screenshots/50 bounces - 1.png}
    \caption{Visualizing ray depth. Left: 5 bounces results in black surfaces where light terminates early. Right: 50 bounces enables the "Hall of Mirrors" effect.}
\end{figure}

\subsection{Scene 4: Low-Light Specular Response}
Set in a void with a dark sky, this scene features one central sphere and three dim, colored light sources.
\begin{itemize}
    \item \textbf{Objective}: Verify the photometric accuracy of light reflection on mirrors.
    \item \textbf{Observation}: The mirror-like sphere correctly picks up the three distinct light sources, proving that the bounce logic correctly handles light transport even in low-energy environments.
\end{itemize}

\begin{figure}[h]
    \centering
    \includegraphics[width=0.7\textwidth]{screenshots/Mirror Lighting.png}
    \caption{Testing specular reflection under colored light sources. The sphere correctly reflects the discrete light positions.}
\end{figure}

\subsection{Scene 5: The Ocean Scene}
The most complex scene, combining vertex animation, dual-layer transparency, and a high-intensity emissive sun near the horizon.
\begin{itemize}
    \item \textbf{Objective}: Stress-test the hybrid nature of the engine (Animation + Ray Tracing).
    \item \textbf{Optical Complexity}: This scene brings to light the difficulty of the glare approximation. Managing the glare intensity so that it reflects in the moving waves without "bleeding" through the objects is the primary challenge here, and it still isn't resolved since it would ask for me to do ray marching.
\end{itemize}

\begin{figure}[h]
    \centering
    \includegraphics[width=0.8\textwidth]{screenshots/Sea.png}
    \caption{Final result of the Sea Scene, featuring animated waves, reflections of the glare, and dual-layer water depth.}
\end{figure}

\section{Performance Analysis and Results}
The performance of the path tracer is primarily limited by the number of samples per pixel (SPP) and the recursion depth (bounces).

\begin{table}[h!]
\centering
\begin{tabular}{@{}lccc@{}}
\toprule
\textbf{Configuration} & \textbf{Samples} & \textbf{Max Bounces} & \textbf{Frame Time (ms)} \\ \midrule
Draft Mode             & 1                & 2                    & 2.2                      \\
Interactive            & 4                & 6                    & 7.5                      \\
High Quality           & 16               & 12                   & 29.8                     \\ \bottomrule
\end{tabular}
\caption{Rendering performance on a high-end Linux desktop (Scene 3 - Mirror Scene).}
\end{table}

The "Firefly" artifacting (isolated bright pixels) was a recurring issue when dealing with high-intensity emissive sources. This was partially mitigated using a clamping threshold on the sample contribution, though at the expense of energy conservation.

\begin{thebibliography}{9}
\bibitem{pbrt} Pharr, M., Jakob, W., and Humphreys, G. (2016). \textit{Physically Based Rendering: From Theory To Implementation}. Morgan Kaufmann.
\bibitem{baraff} Baraff, D. (1997). \textit{An Introduction to Rigid Body Simulation}. SIGGRAPH Course Notes.
\bibitem{brian} Brian Vincent Mirtich (1989). \textit{Impulse based Dynamic Simulation of Rigid Body Systems}.
\end{thebibliography}

\end{document}
